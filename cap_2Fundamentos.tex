
\chapter{Fundamentação Teórica\label{chap:Fundamentacao}}

% Resumo opcional. Comentar se não usar.
%\resumodocapitulo{Resumo}

\section{Tecnologias comumente utilizadas}
 
 Existem diversos modos de estimar a ocupação de um certo ambiente por pessoas. Sensores de presença infravermelhos passivos já são amplamente utilizados comercialmente em aparelhos de ar-condicionado, especialmente modelos split para uso em pequenas salas.
 
 Sensores passivos de presença, entretanto, não permitem um controle mais fino da atuação do ar-condicionado. Eles auxiliam na economia de energia indicando a presença ou não de alguma pessoa ou animal no ambiente, mas sem indicar quantas pessoas etão no local ou quais são suas características. Além disso estão muito sujeitos a interferências, causando falsos positivos ou falsos negativos, devido ao modo de atuação passivo que apenas verifica a emissão de radiação infravermelha em comprimentos de onda próximos ao dos humanos.

 Uma alternativa possível é o uso de RFID. Esta é uma tecnologia que torna-se mais eficiente, acessível e popular a cada dia. O monitoramento posicional de objetos dentro de uma loja ou armazém tornou-se uma prática possível com RFID passivo, prática que é aplicada com soluções comerciais disponíveis no mercado.
 
\section{RFID} \label{section: RFID}

A comunicação do tipo Identificação por Radiofrequência – \textit{Radio Frequency Identification} (RFID) – é um tipo de comunicação sem fio por sinais de rádio. O RFID é uma alternativa de identificação útil em aplicações que exigem a leitura de grandes quantidades de dados \cite{rao1999overview}. 

A composição geral do RFID é feita por uma \textit{TAG} ou etiqueta e uma leitora (reader) que são incorporadas com um transmissor e um receptor no computador \cite{EPC-RFID-link}.

\begin{figure}[H]
    \centering
    \includegraphics[width=0.6\linewidth]{figs/Fundamentos/Composicao.png}
    \caption{\textcolor{red}{Composição básica de um sistema RFID. (Adaptado de EPC-RFID.Info \cite{EPC-RFID-link})}}
    \label{fig:ComposicaoRFID}
\end{figure}


As primeiras aplicações comerciais de RFID, como artigo eletrônico de vigilância (Electronic Article Surveillance), aconteceram no final da década de 1960, desenvolvidas por empresas como: Kongo, Sensormatic (by Johnson Controls) e Checkpoint \cite{chawla2007overview}.

De acordo com Chawla e Ha \cite{chawla2007overview}, o sistema RFID consiste em leitoras e TAGS. A leitora tem como função se comunicar na faixa de alcance sem fio e coletar informações sobre os objetos aos quais as TAGS estão anexadas (ex.: entrada e saída de pessoas de um estabelecimento).

\begin{figure}[H]
    \centering
    \includegraphics[width=0.6\linewidth]{figs/Fundamentos/RFIDdiagram.png}
    \caption{Diagrama de rede de uma comunicação RFID. (Adaptado de Landt, Jeremy \cite{landt2005history}}
    \label{fig:DiagramaRFID}
\end{figure}

\begin{figure}[H]
    \centering
    \includegraphics[width=0.8\linewidth]{figs/Fundamentos/RFIDdetails.png}
    \caption{\textcolor{red}{Composição básica de um sistema RFID. (Adaptado de Chawla e Ha \cite{chawla2007overview})}}
    \label{fig:DetalhesRFID}
\end{figure}



Vogt \cite{vogt2002multiple}, em seu trabalho, investigou a aplicabilidade da RFID passiva na identificação de múltiplos objetos.

\begin{figure}[H]
    \centering
    \includegraphics[width=0.6\linewidth]{figs/Fundamentos/RFID.png}
    \caption{\textcolor{red}{RFID - . (Adaptado de Chen et al. \cite{chenUsingRFID})}}
    \label{fig:RFID-Middleware}
\end{figure}


\subsection{RAIN RFID e IoT}

O padrão RAIN (acrônimo para \textit{RAdio frequency IdentificatioN} - Identificação por radiofrequência) é a união de diferentes tipos de hardware e software que possuem o propósito de ligar o meio RFID UHF com a computação em nuvem. É padronizada pela organização global de mesmo nome (RAIN) com o objetivo de popularizar o uso de RFID. \cite{RAIN}

RAIN RFID foi desenvolvido principalmente para aplicações de internet das coisas (IoT - \textit{Internet of things} em inglês). IoT é, segundo Gubbi, et. al. \cite{gubbi2013internet}:

\begin{quote}
    "Interconexão de dispositivos de detecção e atuação, oferecendo a capacidade de compartilhar informações entre plataformas por meio de uma estrutura unificada, desenvolvendo uma imagem operacional comum para permitir aplicativos inovadores. Isso é conseguido através da detecção onipresente e contínua, análise de dados e representação de informações com a computação em nuvem como a estrutura unificadora."
\end{quote}

\subsection{\textit{TAGs}}
TAGs, também conhecidas como transponders, são etiquetas ou chips microprocessadores de identificação. Esses dispositivos recebem sinal de rádio e transmitem, automaticamente, um sinal diferente. Cada TAG possui um número serial de identificação que a difere de outras. \cite{chawla2007overview}. Esses objetos podem armazenar informações como \textit{serial number} (número serial), \textit{model number} (número do modelo), bem como outras características como: data, cor, tamanho e preço. A TAG RFID é classificada de acordo com sua memória, tipo de comunicação e uso de energia, como é possível visualizar na tabela \ref{tab:comparativoTags}  \cite{AhmedIntegrationStreamMapping}.

 
\begin{table}[H]
\centering
\caption{Classificação de TAGs RFID (Adaptado de \cite{AhmedIntegrationStreamMapping})}
\label{tab:comparativoTags}
\begin{tabular}{p{4cm}|p{6cm}}
\hline
\multicolumn{2}{c}{\cellcolor{lightgray}{Origem da alimentação elétrica}} \\ \hline
Passiva         &   Alimentada pela leitora através do próprio sinal de radiofrequência. A energia do sinal capturada pela antena da TAG alimenta um         \\ \hline
Semi-Passiva    &   text    \\ \hline
Ativa           &   text    \\ \hline
\multicolumn{2}{c}{\cellcolor{lightgray}{Título}} \\ \hline
1       &   2       \\ \hline
data    &   text    \\ \hline
\multicolumn{2}{c}{\cellcolor{lightgray}{Título}} \\ \hline
1       &   2       \\ \hline
data    &   text    \\ \hline
\end{tabular}
\end{table}



É possível classificar as TAGs em três tipos, como pode ser visto na tabela \ref{tab:comparativoTags}: ativas, passivas e semi-passivas \cite{chawla2007overview}.  Existem vários tipos de TAGS ativas (alimentadas por baterias) e passivas, em várias faixas de frequência, presentes no mercado \cite{rao1999overview}. A Figura \textcolor{red}{NUMERO DA FIGURA}, logo abaixo, mostra exemplos de TAGS ativas e passivas.

\textcolor{red}{fotos da tag ativa}

\textcolor{red}{fotos da tag passiva}

Segundo Rao \cite{rao1999overview}, TAGS ativas, como as apresentadas na Figura 1A, tem como vantagens o fornecimento de faixas de leituras maiores (explicar o porquê isso é interessante). Como desvantagem, elas podem ser bem mais caras por exigirem o uso de baterias \cite{rao1999overview}. Uma etiqueta RFID passiva, por sua vez, usará a energia das ondas de rádio do interrogador para retransmitir suas informações armazenadas de volta ao interrogador \cite{EPC-RFID-link}. 




\subsection{Antenas}

A antena é capaz de perceber o sinal de uma TAG dentro de um determinado raio de cobertura. Cada antena é única nesse sentido...


\subsection{Efeitos causados por materiais e frequência de operação}

É amplamente conhecido na indústria do RFID e de outras tecnologias que trabalham com RF (radiofrequência - \textit{radiofrequency} em inglês) que os materiais presentes no ambiente onde se propagam os sinais transmitidos. Por exemplo, é praticamente um consenso o fato de que os meios aquosos absorvem os sinais, e os metais os refletem.

Estudos com materiais mostram que podem existir nove tipos de perdas no sinal, como mostra Sanghera em seu livro\cite{book:SangheraRFID+}:
\begin{itemize}
    \item \textbf{Absorção:} Energia do sinal absorvida pelo meio;
    \item \textbf{Atenuação:} Diminuição da amplitude do sinal devido à absorção e dispersão;
    \item \textbf{Efeitos Dielétricos:} Capacidade do meio de reter carga. Gera atraso do sinal;
    \item \textbf{Difração:} "Dobra" de um sinal de radiofrequência ao atingir uma quina ou pequeno orifício;
    \item \textbf{Perda de espaço livre:} Diminuição da densidade do sinal devido à característica de propagação do sinal para todos os lados;
    \item \textbf{Interferência:} Ocorre no encontro de duas ondas de radiofrequência, pode ser construtiva ou destrutiva;
    \item \textbf{Reflexão:} Retorno do sinal em um ângulo isóceles ao de incidência em uma superfície plana. Geralmente ocorre no chão, teto e paredes;
    \item \textbf{Refração:} mudança de direção de trajetória do sinal, entretanto permeando o meio causador da refração;
    \item \textbf{Espalhamento:} Absorção e re-emissão do sinal, em direções aleatórias.
\end{itemize}
    
Sanghera mostra ainda uma tabela contendo a influência de cada material no sinal de radiofrequência \cite{book:SangheraRFID+}. Esta tabela é adaptada e apresentada a seguir (Tabela \ref{tab:InterfMateriais}).

\begin{table}[H]
\centering
\caption{Efeito dos materiais nos sinais de RFID (Adaptado de \cite{book:SangheraRFID+})}
\label{tab:InterfMateriais}
\begin{tabular}{p{5cm} p{7cm}}
\hline
\cellcolor{lightgray}{Material} & \cellcolor{lightgray}{Efeito no sinal RF}   \\ \hline
Papelão         &   Absorção        \\
Líquido condutivo    &   Absorção        \\
Vidro           &  Atenuação        \\
Conjunto de latas & Reflexão em múltiplas direções \\
Corpo humano ou animal & Absorção, dessintonização e reflexão \\ 
Metal & Reflexão \\
Plástico & Dessintonização devido ao efeito dielétrico \\ \hline
\end{tabular}
\end{table}

 Um estudo conduzido por Fletcher \cite{fletcher2005study} questiona, contudo, os efeitos dos sinais RF nos meios aquosos, e portanto, os efeitos no corpo humano e de animais por consequência. Ele afirma que, ao contrário do que se acredita, os líquidos interferem nas leituras dos sinais de RFID majoritariamente devido aos efeitos de reflexão e refração, enquanto o consenso é de que meios aquosos absorviam os sinais, principalmente.
 
 De uma forma ou de outra, as leituras RFID obtidas são de baixa qualidade quando os transpônderes ou as antenas estão próximas de meios aquosos ou pessoas e animais. Esse é o principal motivo de o monitoramento de localização por RFID, ou outras tecnologias RF, não possuir muita aderência para o propósito de monitorar pessoas enquanto a tecnologia avança a passos largos no monitoramento de produtos e aplicações de \textit{IoT}, visto que estes últimos são menos suscetíveis à perda de sinal.
 
 O livro de Thornton e Sanghera \cite{thornton2011cheat} mostra a influência causada pela escolha da frequência de funcionamento do sistema. Os protocolos RFID preveem o funcionamento da tecnologia em quatro faixas de frequência:
 
 \begin{itemize}
     \item Baixa frequência (LF - \textit{Low Frequency})
     \begin{itemize}
         \item opera a 125 kHz ou 134 kHz para dispositivos RFID
         \item Baixa distorção devido a água
         \item Curto alcance (até 10 cm)
         \item Utilizado para identificação de pessoas e animais à curta distância
     \end{itemize}
     \item Alta frequência (HF - \textit{High Frequency})
     \begin{itemize}
         \item opera a 13.56 MHz para dispositivos RFID
         \item alcance de até 3m
         \item sinal RF com comprimento de onda pequeno / baixa capacidade de penetração em materiais
         \item mais opções de velocidade de leitura em relação à baixa frequência
         \item utilizado em controle de acesso e rastreio de itens como bagagens
     \end{itemize}
     \item UHF (em inglês \textit{Ultra High Frequency} - Frequência Ultra Alta em tradução livre para o português)
     \begin{itemize}
         \item Diferentes frequências em diferentes países
         \begin{itemize}
             \item Faixa de 860-960MHz
             \item 902-928MHz no Brasil (geralmente 915MHz)
         \end{itemize}
         \item alta distorção devido a água
         \item a frequência de operação depende do país onde o equipamento é homologado
         \item maior alcance que LF e HF
         \item usado para gerenciamento de estoque e armazéns
     \end{itemize}
     \item Microondas
     \begin{itemize}
         \item Faixa de 2,44-5,80GHz para dispositivos RFID
         \item Alta velocidade de leitura
         \item Longo alcance
         \item baixo desempenho próximo a água
         \item utilizado para identificação de veículos e \textit{supply chain} (em português, "cadeia de suprimentos" ou "cadeia de logística")
     \end{itemize}
 \end{itemize}


A imagem \ref{fig:impactoFrec} de Oliveira e Rocha \cite{TG2013OliveiraERocha}, traduzida da original no livro de Thornton e Sanghera \cite{thornton2011cheat}, simplifica os conceitos do impacto da frequência de operação no sinal RFID.

\begin{figure}[H]
    \centering
    \includegraphics[width=0.8\linewidth]{figs/Fundamentos/frequencia.png}
    \caption{Impacto da frequência de operação nas leituras RFID. (Fonte: Oliveira e Rocha \cite{TG2013OliveiraERocha}, adaptado de Thornton e Sanghera \cite{thornton2011cheat})}
    \label{fig:impactoFrec}
\end{figure}
    
    
\section{Localização \textit{indoor}}

O rastreio e localização de pessoas em ambientes fechados é algo fortemente desejado desde o surgimento das tecnologias de localização por sinais de radiofrequência. É imaginado um futuro onde equipes de segurança sabem exatamente onde resgatar uma pessoa com dificuldades, dentro de um edifício; sistemas de ar-condicionado eficientes dimensionam automaticamente as suas potências baseados na ocupação do local aumentando a eficiência energética; mapas de ambientes fechados públicos que indicam o caminho até o destino final interno.

A localização de pessoas e objetos já está muito avançada para ambientes externos, com exemplos de muito sucesso como o sistema de posicionamento global - GPS (\textit{Global Positioning System} em inglês) e seus similares (Galileo, Glonass, Compass), e também do Cell-ID - tecnologia de localização de dispositivos GSM/WCDMA/CDMA por triangulação de torres de celular.

Contudo, a localização em ambientes fechados ainda é muito restrita, principalmente devido às restrições de comunicação por sinais RF em ambientes passíveis de obstrução por objetos e fontes de reflexão e demais tipos de interferência. Problemas como a atenuação e dispersão do sinal, sinais duplos ou múltiplos, e a grande quantidade de dispositivos comunicando-se em um pequeno espaço em algumas aplicações (o que gera conflito de diversas tentativas de estabelecimento de comunicação ao mesmo tempo) dificultam o desenvolvimento de uma tecnologia definitiva para esta aplicação.

Para desviar dos empecilho, soluções específicas para diversas aplicações são criadas. A indústria larga na frente, com sistemas de rastreio e posicionamento de produtos e equipamentos, como em armazéns e linhas de produção. O comércio também já possui soluções para prevenção de perdas e roubos, e começa a aplicar os conceitos de rastreamento de produtos de armazéns para identificar roupas e sapatos em estantes, por exemplo. A área de soluções para cidades pesquisa tecnologias IoT.

A área de localização de pessoas possui 

\subsection{Estratégias}

