
\chapter{Conclusões} \label{chap:Conclusao}

\label{CapConclusoes}

Os resultados obtidos foram promissores para os critérios de decisão de comparação de picos de potência de sinal RSSI e de comparação do último valor de RSSI. Ambos obtiveram eficiência observada de $80\%$. Estes resultados poderiam ser muito mais altos, pois 129 amostragens de leitura e 10 transições podem ter sido um teste muito pequeno para validar estatisticamente. Dessa forma, novos testes com mais dados podem ser conduzidos.

Os resultados obtidos com o critério de transição por comparação de frequência Doppler foram abaixo do esperado, dado o resultado surpreendente do teste preliminar da sessão \ref{sect:testepreliminar}. Esperava-se uma eficiência maior que $70\%$ de acerto, entretanto, este modelo obteve apenas $60\%$ de acerto. Como há evidência de que este método pode funcionar com boa taxa de acerto no teste preliminar, provavelmente este método pode ser corrigido para obter maior eficiência.

O resultado obtido com a união do critério de comparação de picos e de efeito Doppler teve rendimento muito abaixo do esperado, com apenas $50\%$ de eficiência. Isso provavelmente se deve ao fato de que ele necessita que tanto o pico de RSSI quanto a transição de valores positivos para negativos do Efeito Doppler aconteçam. Caso a leitora perca um dos eventos, este método não contabiliza a mudança de ambiente. Este método pode ser flexibilizado para aceitar ambos os critérios com lógica "OU" ao invés de lógica "E". Isso permitiria, talvez, uma taxa de acerto muito maior do que os outros métodos.

Os testes de mediana e média foram feitos sem muitas expectativas, e renderam resultados interessantes. O critério de média obteve resultado oposto, o que indica que ele funciona de forma inversa, e que, se o critério de decisão for ajustado de forma a a tomar a decisão oposta, este pode se tornar um método válido. Ele não se apresenta um método robusto, entretanto, possuindo muita variação. O critério de mediana obteve eficiência de $60\%$, o que já era esperado.

Porém, para a aplicação motivadora deste trabalho - o controle de sistemas de ar-condicionado com foco em economia de energia - questiona-se se em condições reais os custos inviabilizem a aplicação da a técnica desenvolvida. Os custos estimados da compra e instalação dos equipamentos RFID UHF, o consumo energético de cada leitora e a compra e distribuição de TAGs, podem tornar esta uma abordagem pior do que alternativas mais baratas, como a utilização de sistemas de controle de acesso existentes, sistemas de circuito fechado de televisão com processamento de imagens ou sensores de presença infravermelhos ou de feixe de laser.

Entretanto, para fins acadêmicos e de desenvolvimento tecnológico a aplicação é válida, pois possui potencial para aprimoramento e possui diversas aplicações possíveis diferentes. Até mesmo para a aplicação imaginada de controle de sistemas HVAC específicos como áreas de grande movimentação de pessoas (como estações de metro - onde todos já possuem um bilhete com etiqueta RFID embutida), ou locais onde o controle acurado de temperatura é o critério mais importante, como um \textit{datacenter} ou laboratório, esta abordagem pode ser implementada com sucesso.


\section{Perspectivas Futuras}

Este trabalho inovou nos métodos de identificação de mudança de ambiente em relação aos seus antecessores \cite{TG2013OliveiraERocha} \cite{TG2015RaissaERenata}. A nova abordagem na aplicação e nos algoritmos abre espaço para a aplicação de novas técnicas. Em especial destacam-se a lógica difusa, ou lógica \textit{fuzzy} \cite{yen1999fuzzy}, e a implementação de filtros preditivos, como o filtro de Kalman \cite{welch1995introduction}.

A lógica \textit{fuzzy} pode ser usada para aprimoramento do critério de decisão sobre em qual lado da porta ou passagem o portador da TAG se encontra. Já os filtros preditivos, seja o filtro de Kalman ou outro, podem ser úteis na hora de coletar os dados para obter leituras mais precisas e acuradas, tornando o processo de decisão mais confiável.

É aconselhado também buscar outras abordagens, como o LANDMARK \cite{bekkali2007rfid}, que tem o intuito de criar um mapa do ambiente em que se deseja monitorar as pessoas, e por meio dos padrões de magnitude dos sinais estima o local exato dentro de uma sala onde a pessoa se encontra.
