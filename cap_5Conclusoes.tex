
\chapter{Conclusões} \label{chap:Conclusao}

\label{CapConclusoes}

Os resultados obtidos foram promissores para os critérios de decisão de comparação de picos de potência de sinal RSSI e de comparação do último valor de RSSI. Ambos obtiveram eficiência observada de $100\%$. Estes resultados, apesar de muito bons, podem ser validados de forma melhor com uma amostra de testes maior do que 10 transições. Dessa forma, novos testes com mais dados podem ser conduzidos.

Os resultados obtidos com o critério de transição por comparação de frequência Doppler foram abaixo do esperado, dado o resultado surpreendente do teste preliminar da sessão \ref{sect:testepreliminar}. Esperava-se uma eficiência maior que $80\%$ de acerto, entretanto, este modelo obteve apenas $70\%$ de acerto. Como há evidência de que este método pode funcionar com boa taxa de acerto no teste preliminar, provavelmente este método pode ser corrigido para obter maior eficiência.

O resultado obtido com a união do critério de comparação de picos e de efeito Doppler teve rendimento abaixo do esperado, com apenas $70\%$ de eficiência, nivelando a eficiência dos dois métodos dos quais é composto pelo pior caso. Isso provavelmente se deve ao fato de que ele necessita que tanto o pico de RSSI quanto a transição de valores positivos para negativos do Efeito Doppler aconteçam ao mesmo tempo para registrar uma transição. Caso a leitora perca um dos eventos, este método não contabiliza a mudança de ambiente. Este método pode ser flexibilizado para aceitar ambos os critérios com lógica "OU" ao invés de lógica "E". Isso permitiria, talvez, uma taxa de acerto muito maior do que os outros métodos.

Os testes de mediana e média foram feitos sem muitas expectativas, e renderam resultados pouco expressivos - ambos obtiveram rendimento abaixo de 50\%. Estes dois métodos são muito lentos para a velocidade média com que uma pessoa atravessa uma porta. Eles possuem potencial, entretanto, como métodos de correção para os outros, em casos onde todo o ambiente possui cobertura de sinal pelas antenas, pois sua natureza lenta acumula leituras repetidas durante longos períodos de tempo, e podem ser utilizados para detectar uma pessoa parada no interior de uma sala.

Este trabalho possui grande potencial de implementação em sistemas reais de ERP,EPC, especialmente em ambientes diferenciados, tais como a área cirúrgico-laboratorial. Esse tipo de ambiente exige um controle fino da temperatura, integrando inteligência artificial para controle preditivo, para que os \textit{setpoints} de condições ambientais sejam sempre atingidos em tempo mínimo e com a menor variação possível. Além disso, exigem um severo sistema se segurança e controle de acesso, onde o uso das mãos é indesejado por se tratar de ambientes com risco biológico envolvido.

Considerando os fins acadêmicos e de desenvolvimento tecnológico, o trabalho desenvolvido já possui grande valor, pois possui potencial para aprimoramento e possui diversas aplicações possíveis diferentes. Dado o sucesso dos resultados obtidos com os métodos de comparação do último valor RSSI e dos tempos de picos de potência do sinal de retorno, sistemas de ar condicionado específicos podem ser controlados utilizando dados gerados por aplicações como o programa desenvolvido. Alguns exemplos de aplicações diferentes do setor hospitalar são áreas de grande movimentação de pessoas (como estações de metrô - onde todos já possuem um bilhete com etiqueta RFID passiva embutida), ou locais onde o controle acurado de temperatura é o critério mais importante, como um \textit{datacenter} ou laboratório de química.


\section{Perspectivas Futuras}

Este trabalho inovou nos métodos de identificação de mudança de ambiente em relação aos seus antecessores \cite{TG2013OliveiraERocha} \cite{TG2015RaissaERenata}. A nova abordagem na aplicação e nos algoritmos abre espaço para a aplicação de novas técnicas. Em especial destacam-se a lógica difusa, ou lógica \textit{fuzzy} \cite{yen1999fuzzy}, e a implementação de filtros preditivos, como o filtro de Kalman \cite{welch1995introduction}.

A lógica \textit{fuzzy} pode ser usada para aprimoramento do critério de decisão sobre em qual lado da porta ou passagem o portador da TAG se encontra. Já os filtros preditivos, seja o filtro de Kalman ou outro, podem ser úteis na hora de coletar os dados para obter leituras mais precisas e acuradas, tornando o processo de decisão mais confiável.

É aconselhado também buscar outras abordagens, como o LANDMARK \cite{bekkali2007rfid}, que tem o intuito de criar um mapa do ambiente em que se deseja monitorar as pessoas, e por meio dos padrões de magnitude dos sinais estima o local exato dentro de uma sala onde a pessoa se encontra.

Para finalizar, o desenvolvimento de uma integração com sistemas ERP,EPC é altamente recomendado, para que os estimadores de localização e contabilização da quantidade de pessoas presentes possam ser integrados em sistemas supervisórios comerciais, sistemas de controle de ar condicionado e sistemas de controle de acesso.