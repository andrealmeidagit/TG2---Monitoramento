
\chapter{Introdução}

\label{CapIntro}

% Resumo opcional. Comentar se não usar.
\resumodocapitulo{O desenvolvimento tecnológico culminou na demanda por melhorias de serviços que, outrora, não eram capazes de integrar os fatores aliados à facilidade e ao bem-estar dos usuários. O bem-estar, dentro de edifícios, está associado diretamente ao conforto ambiental. Esse conceito é estudado nos ramos de arquitetura e construção civil e visa a garantia das condições que possibilitam a qualidade física e psicológica dos indivíduos dentro dos mais diversos ambientes e instalações.}

\section{Contextualização}

O momento em que se elabora este projeto é uma época de grandes mudanças em quesitos de tecnologia. O entendimento do impacto ao meio ambiente, gerado pelo consumo energético em residências, estabelecimentos comerciais e fabris começa a ser compreendido por boa parte das pessoas, e principalmente o conhecimento da escalabilidade do impacto individual das pessoas de uma população.

Nas últimas décadas, a racionalização do consumo energético vem sendo alvo de diversos encontros técnicos da área de sustentabilidade, e de diversos acordos globais para a redução das emissões de carbono, geradas principalmente, devido à ação humana, especialmente na produção de energia.

Ao mesmo tempo em que o consumo é questionado, as exigências quanto à qualidade de vida, segurança no trabalho, ergonomia e conforto passaram a ser aspectos chave na vida e no dia-a-dia das pessoas. Seja no trabalho, ou no ambiente domiciliar, o conforto térmico é decisivo para o bem-estar das pessoas, assim como para a produtividade no ambiente de trabalho.

É um desafio alcançar a completude do conceito de conforto ambiental nas edificações. Nesse sentido, a automação é capaz de otimizar situações de uso de instrumentos cotidianos para vantagens como: a facilitação de trabalhos, muitas vezes manuais, para a economia de energia, para o estabelecimento de relações automáticas entre duas variáveis, a exemplo da entrada de pessoas em uma sala e o acionamento do ar condicionado, dentre outras vantagens.

\subsection{Conforto térmico}

Conforto Ambiental é um conjunto de condições que possibilitam o bem-estar físico e psicológico dos indivíduos em diferentes locais, sendo que, o enfoque deste trabalho é principalmente em relação ao interior de instalações.

% É um desafio, para todas as pessoas responsáveis por um projeto, alcançar a completude do conceito de conforto ambiental em edificações.

Esse conforto pode ser subdividido em categorias como: visual (abrangendo iluminação e estética visual), ergonômico (que tem muito destaque na parte de segurança do trabalho e com exemplos claros de esforços repetitivos, etc.), acústico (onde há limites em decibéis muito bem definidos) e térmico, que se configura como objeto desse trabalho. As normas que regem a construção e habitação de ambientes quanto ao aspecto de conforto ambiental e térmico são as NBRs 15.575 \cite{NBR15575} e 15.220 \cite{NBR15220}.

\subsection{Tecnologias de comunicação sem fio}

O uso de RFID, Bluetooth, Zigbee pode auxiliar na automação e possibilitar a instalação de sistemas de condicionamento baseados na movimentação (entrada e saída) de pessoas. Segundo Ahmed et al. \cite{AhmedIntegrationStreamMapping}, sistemas wireless (sem fio) podem reduzir significativamente a frequência de erros humanos, otimizar a gestão e aumentar a acurácia da informação principalmente em relação à indoor network (rede interna). Chen et al. \cite{chenUsingRFID}, retrata ainda, quão importante é a utilização de ferramentas como o RFID para a rastreabilidade e compra segura de alimentos. O uso de RFID pode ser empregado até no controle do fluxo de saída de roupas de uma loja.

Esses sistemas que funcionam à distância, sem fio, como o RFID, que é o sistema de identificação por radiofreqüência (RFID), que será melhor explicado no capítulo \ref{chap:EstOcpAmb}, permitem recuperar informações armazenadas de um objeto preso ou incorporado a bens, produtos ou seres vivos \cite{gutierrez2005complexo}. Dessa forma, o emprego de tecnologias sem fio pode proporcionar a melhor administração de indivíduos em situações rotineiras e também em situações de emergência. 

\section{Proposta}

A proposta deste trabalho consiste em utilizar a tecnologia RFID passivo para estimar a quantidade de pessoas que ocupam um determinado ambiente. O cenário vislumbrado é de um edifício de escritórios, sala de aula, anfiteatro ou auditório, oficina ou fábrica, ou qualquer ambiente cuja definição exata da quantidade de pessoas que ocupam tal ambiente seja relevante.

A motivação principal para a elaboração do trabalho foi a estimação do impacto na carga térmica causado pelas pessoas que transitam e ocupam um ambiente fechado. O princípio almejado é realizar o controle de equipamentos de ventilação, aquecimento e ar-condicionado (HVAC) de forma antecipada à variação das condições climáticas internas, em especial a temperatura e a umidade. Este propósito é considerado importante para sistemas de ar-condicionado simples e complexos, seja para manutenção do conforto térmico em um ambiente de trabalho, seja para minimizar a variação de temperatura em um ambiente em as condições do ar são críticas, como uma sala de computadores e servidores.

Este propósito pode ser extrapolado para aplicações de controle de acesso e segurança, como por exemplo, a definição de alarmes de intrusão de áreas restritas, alarmes de quantidade máxima de ocupantes por sala, ou rastreamento da trajetória de visitantes em um ambiente. Utiliza-se como exemplo um laboratório onde, para preservar as condições de limpeza, segurança e temperatura, limita-se o número de pessoas que pode transitar dentro deste ambiente por vez.

\section{Trabalhos anteriores}

A elaboração deste projeto dá continuidade à uma sequência de trabalhos executados no Laboratório de Automação e Robótica da Universidade de Brasília (LARA - UnB). Em especial destacam-se o trabalho de Oliveira, Filipe e Rocha, Frederico \cite{TG2013OliveiraERocha} e o trabalho de Alves, Raissa e Chupel, Renata \cite{TG2015RaissaERenata}, este último a partir do qual dá-se continuidade.
Este trabalho estima obter uma nova abordagem aos equipamentos de RFID disponíveis no laboratório, para obter a identificação da quantidade de pessoas em um ambiente, através de um novo algoritmo, e uma nova disposição física dos equipamentos no laboratório.

\section{Objetivos}

O principal objetivo deste trabalho é criar um espaço inteligente, que utiliza dados de tecnologias sem fio para aprimorar a qualidade de vida e de trabalho das pessoas, ao mesmo tempo em que proporciona economia de energia e segurança.

O objetivo específico do trabalho é desenvolver uma aplicação RAIN RFID para monitorar o trânsito de pessoas em um ambiente fechado e contabilizar a quantidade de pessoas presentes em cada subdivisão deste ambiente, em tempo real.